\documentclass[10pt,a4paper]{article}
\usepackage[UTF8]{ctex}
\usepackage{xcolor}
\usepackage{titlesec}
\usepackage{enumitem}
\usepackage{fontawesome5}
\usepackage{geometry}
\usepackage{tikz}
\usepackage{background}
\usepackage{multicol}
\geometry{margin=1.2cm}

% 颜色定义
\definecolor{primary}{RGB}{62, 80, 200}
\definecolor{accent}{RGB}{52, 152, 219}
\definecolor{lightgray}{RGB}{240, 240, 240} % 浅灰色
\definecolor{mediumgray}{RGB}{200, 200, 200} % 中灰色
\definecolor{darkgray}{RGB}{160, 160, 160} % 深灰色(或称灰色)

% 设置背景
\backgroundsetup{
contents={
\begin{tikzpicture}[remember picture,overlay]
    % 左侧浅灰色背景
    \fill[lightgray] (current page.north west) rectangle ([xshift=0.333\paperwidth]current page.south west);
    % 中间中灰色背景
    \fill[mediumgray] ([xshift=0.333\paperwidth]current page.north west) rectangle ([xshift=0.666\paperwidth]current page.south west);
    % 右侧深灰色背景
    \fill[darkgray] ([xshift=0.666\paperwidth]current page.north west) rectangle (current page.south east);
\end{tikzpicture}
},
scale=1,
angle=0,
opacity=1
}

% 紧凑的章节格式
\titleformat{\section}{\large\bfseries\color{primary}}{}{0em}{}
\titlespacing*{\section}{0pt}{8pt}{4pt}

% 自定义环境 - 更紧凑
\newenvironment{resumesection}[1]{
    \section*{\color{primary} #1}
    \vspace{-6pt}
    \hrule
    \vspace{6pt}
}{}

% 紧凑的进度条
\newcommand{\skillbar}[1]{
    \begin{tikzpicture}[scale=0.8]
        \draw[fill=white, rounded corners=1pt] (0,0) rectangle (3.5,0.3);
        \draw[fill=accent, rounded corners=1pt] (0,0) rectangle (#1*0.175,0.3);
    \end{tikzpicture}
}

\begin{document}

% 紧凑的头部信息
\begin{center}
    {\LARGE\bfseries\color{primary} 张三} \\
    \vspace{3pt}
    {\large\color{accent} 高级软件工程师} \\
    \vspace{8pt}
    \footnotesize
    \begin{tabular}{ccc}
        \faPhone\ 138-0013-8000 & 
        \faEnvelope\ zhangsan@email.com & 
        \faMapMarker\ 北京市海淀区
    \end{tabular}
\end{center}

\vspace{12pt}

\begin{minipage}[t]{0.28\textwidth}
    \raggedright
    \footnotesize
    
    % 联系方式
    \begin{resumesection}{联系方式}
        \begin{itemize}[leftmargin=*,nosep,itemsep=2pt]
            \item \faPhone\ 138-0013-8000
            \item \faEnvelope\ zhangsan@email.com
            \item \faMapMarker\ 北京市海淀区
            \item \faGithub\ github.com/zhangsan
        \end{itemize}
    \end{resumesection}

    % 技能专长 - 更紧凑
    \begin{resumesection}{技能专长}
        \textbf{编程语言} \\
        Java \skillbar{18} \\
        Python \skillbar{16} \\
        JavaScript \skillbar{15} \\
        SQL \skillbar{14} \\
        C++ \skillbar{13} \\
        
        \vspace{4pt}
        \textbf{框架技术} \\
        Spring Boot \skillbar{17} \\
        React \skillbar{15} \\
        Docker \skillbar{14} \\
        Kubernetes \skillbar{13} \\
        MyBatis \skillbar{16} \\
        
        \vspace{4pt}
        \textbf{开发工具} \\
        Git \skillbar{16} \\
        Maven \skillbar{15} \\
        Jenkins \skillbar{14} \\
        Linux \skillbar{15} \\
        IDEA \skillbar{16} \\
    \end{resumesection}

    % 语言能力
    \begin{resumesection}{语言能力}
        中文 \hfill 母语 \\
        英语 \hfill CET-6 \\
        日语 \hfill N2 \\
    \end{resumesection}

    % 证书荣誉
    \begin{resumesection}{证书荣誉}
        \begin{itemize}[leftmargin=*,nosep,itemsep=2pt]
            \item 阿里巴巴优秀员工
            \item 系统架构设计师
            \item Oracle Java认证
            \item 程序设计竞赛一等奖
        \end{itemize}
    \end{resumesection}
\end{minipage}
\hfill
\begin{minipage}[t]{0.68\textwidth}
    \raggedright
    \footnotesize
    
    % 教育背景
    \begin{resumesection}{教育背景}
        \textbf{计算机科学与技术硕士} \hfill \textit{2018-2021} \\
        清华大学 \hfill GPA: 3.8/4.0 \\
        \vspace{4pt}
        \textbf{软件工程学士} \hfill \textit{2014-2018} \\
        北京邮电大学 \hfill GPA: 3.6/4.0 \\
    \end{resumesection}

    % 工作经历 - 更紧凑
    \begin{resumesection}{工作经历}
        \textbf{高级软件工程师} \hfill \textit{2021-至今} \\
        \textbf{阿里巴巴集团} \hfill 杭州 \\
        \begin{itemize}[leftmargin=*,topsep=2pt,itemsep=1pt,partopsep=0pt]
            \item 负责电商平台核心交易系统开发,日均处理订单量超100万
            \item 主导系统架构优化,将API响应时间从200ms降低至80ms
            \item 设计和实现微服务架构,提高系统可扩展性和稳定性
        \end{itemize}
        
        \vspace{4pt}
        \textbf{软件工程师} \hfill \textit{2019-2021} \\
        \textbf{字节跳动} \hfill 北京 \\
        \begin{itemize}[leftmargin=*,topsep=2pt,itemsep=1pt,partopsep=0pt]
            \item 参与推荐系统后端开发,优化推荐算法准确率提升15\%
            \item 开发高性能数据缓存系统,QPS提升至10万+
            \item 负责代码审查和技术文档编写
        \end{itemize}
        
        \vspace{4pt}
        \textbf{实习软件工程师} \hfill \textit{2018-2019} \\
        \textbf{百度} \hfill 北京 \\
        \begin{itemize}[leftmargin=*,topsep=2pt,itemsep=1pt,partopsep=0pt]
            \item 参与搜索引擎相关模块开发
            \item 学习大型分布式系统开发经验
        \end{itemize}
    \end{resumesection}

    % 项目经验 - 更紧凑
    \begin{resumesection}{项目经验}
        \textbf{智能电商推荐系统} \hfill \textit{2022.03-2022.12} \\
        \begin{itemize}[leftmargin=*,topsep=2pt,itemsep=1pt,partopsep=0pt]
            \item 基于用户行为数据和机器学习算法,构建个性化商品推荐系统
            \item 使用Spring Boot + Redis + MySQL技术栈,实现高并发处理
            \item 项目上线后,用户点击率提升25\%,转化率提升18\%
        \end{itemize}
        
        \vspace{4pt}
        \textbf{分布式缓存中间件} \hfill \textit{2021.06-2021.12} \\
        \begin{itemize}[leftmargin=*,topsep=2pt,itemsep=1pt,partopsep=0pt]
            \item 设计并开发基于Redis的分布式缓存中间件
            \item 实现缓存预热、数据分片、故障自动切换等功能
            \item 支持公司多个业务系统,降低数据库压力60\%
        \end{itemize}
        
        \vspace{4pt}
        \textbf{微服务监控平台} \hfill \textit{2020.09-2021.03} \\
        \begin{itemize}[leftmargin=*,topsep=2pt,itemsep=1pt,partopsep=0pt]
            \item 开发全链路监控系统,集成日志收集、性能监控、告警功能
            \item 使用Prometheus + Grafana + ELK技术栈
            \item 帮助团队快速定位和解决线上问题
        \end{itemize}
    \end{resumesection}

    % 个人总结
    \begin{resumesection}{个人总结}
        5年软件开发和架构设计经验,精通Java和分布式系统开发。\\
        在大型互联网公司有丰富的项目实践经验,擅长系统性能优化\\
        和架构设计,具备团队管理和技术指导能力。
    \end{resumesection}
\end{minipage}

\end{document}